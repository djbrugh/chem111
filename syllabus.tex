%----------------------------------------------------------------------------
% Course Syllabus for Chemistry 111 (c) 2016 by Dale J. Brugh

% Course Syllabus for Chemistry 111 is released under a
% Creative Commons Attribution-ShareAlike 4.0 International License.

% See http://creativecommons.org/licenses/by-sa/4.0/ for
% a description of your rights under this license. 
%----------------------------------------------------------------------------

\documentclass[letterpaper,oneside,onecolumn,11pt,article]{memoir}
%
% --- LOAD PACKAGES ---
%
\usepackage[T1]{fontenc}            % use T1 font encoding
\usepackage{textcomp}
\usepackage{courier}                % set courier as typewriter font
\usepackage{times}                  % set times as text font
\usepackage[scaled=0.92]{helvet}    % set Helvetica as the sans-serif font
\usepackage{mtpro2}
\usepackage{setspace}
\usepackage{amsmath}
\usepackage{graphicx,color}
\usepackage{wallpaper}
\usepackage{textcomp}
\usepackage{relsize,fancyvrb}
\usepackage{verbatim}
\usepackage{caption}
\usepackage{paralist}
\usepackage{enumerate}
\usepackage{boxedminipage}
\usepackage{hyperref}
\usepackage[expansion=true,protrusion=true,kerning=true]{microtype}
%
% --- HYPER SETUP ---
%
\hypersetup{
    unicode=false,          % non-Latin characters in Acrobat’s bookmarks
    pdftoolbar=true,        % show Acrobat’s toolbar?
    pdfmenubar=true,        % show Acrobat’s menu?
    pdffitwindow=true,      % page fit to window when opened
    pdftitle={Course Syllabus: Chemistry 111 / Spring 2016}, 
    pdfauthor={Dale J. Brugh},     % author
    pdfsubject={General Chemistry},   % subject of the document
    pdfnewwindow=true,      % links in new window
    pdfkeywords={classes, ch111s15}, % list of keywords
    colorlinks=true,       % false: boxed links; true: colored links
    linkcolor=black,          % color of internal links
    citecolor=green,        % color of links to bibliography
    filecolor=magenta,      % color of file links
    urlcolor=black           % color of external links
}
\definecolor{nicered}{rgb}{.647,.129,.149}
\definecolor{mutedgrey}{rgb}{0.4,0.4,0.4}
\definecolor{shadecolor}{cmyk}{0,0,0.25,0.07}
\definecolor{MyDarkBlue}{rgb}{0,0.08,0.45}
\definecolor{MarginRed}{rgb}{0.8,0.0,0.0}
\definecolor{MarginBlue}{rgb}{0.2,0.0,1.0}
\definecolor{MarginGrey}{rgb}{0.4,0.4,0.4}
%\renewcommand{\chapnumfont}{\bfseries\Huge\sffamily}
%\renewcommand{\chaptitlefont}{\bfseries\Large\sffamily}
\setsecheadstyle{\bfseries\Large\sffamily\raggedright}
\setsubsecheadstyle{\bfseries\large\sffamily\raggedright}
\setsubsubsecheadstyle{\bfseries\normalsize\sffamily\raggedright}
\renewcommand \thesection{\bfseries\arabic{section}}
\flushbottom
\setstocksize{11in}{8.5in}
%\setlength{\parskip}{5pt}
\settrims{0pt}{0pt}
%\settrimmedsize{11in}{210mm}{*}
%\setlength{\trimtop}{0pt}
%\setlength{\trimedge}{\stockwidth}
%\addtolength{\trimedge}{-\paperwidth}
\settypeblocksize{8.5in}{5.0in}{*}
\setulmargins{1.25in}{*}{*}
\setlrmargins{1.25in}{*}{*}
\setmarginnotes{5mm}{4.0cm}{\onelineskip}
\setheadfoot{\onelineskip}{4\onelineskip}
\setheaderspaces{*}{\onelineskip}{*}
\checkandfixthelayout
%\setlength \fboxsep{0.1in}
\setlength \headwidth{\textwidth+\marginparwidth+\marginparsep}
%
\makepagestyle{courseinformation}
\makerunningwidth{courseinformation}{\headwidth}
\makeheadrule{courseinformation}{\headwidth}{\normalrulethickness}
\makeheadposition{courseinformation}{flushright}{flushleft}{flushleft}{flushleft}
\makeoddhead{courseinformation}%
    {\sffamily Course Syllabus: Chemistry 111 / Spring 2016}{}{\sffamily\thepage}

    %\makeevenfoot{courseinformation}{}{}{\addRevisionData}
    %\makeoddfoot{courseinformation}{}{}{\addRevisionData}

\makepagestyle{courseinformationtitle}
\makerunningwidth{courseinformationtitle}{\headwidth}
\makeheadposition{courseinformationtitle}{flushright}{flushleft}{flushleft}{flushleft}
    %\makeevenfoot{courseinformationtitle}{}{}{\addRevisionData}
    %\makeoddfoot{courseinformationtitle}{}{}{\addRevisionData}

\pagestyle{courseinformation}
\captionsetup{labelsep=colon,aboveskip=0.25cm,justification=RaggedRight,singlelinecheck=false,labelfont={bf,sf}}
%
% --- DEFINE MARGIN FIGURE COMMAND ---
%
\newcommand{\marginfigures}[4]{
\marginpar{\centering
\includegraphics[width=#1]{#2}
\captionsetup{labelsep=newline,aboveskip=-0.5cm,justification=RaggedRight,singlelinecheck=false,labelfont={bf,sf}}
\captionsetup[figure]{position=bottom}
\captionof{figure}{#3}
\label{#4}}%
}%
%
% ---DEFINE MARGIN NOTE COMMAND ---
%
\newcommand{\marginnote}[2]
{%
\marginpar{\raggedright\vspace{#1}\begin{Spacing}{0.65}\sffamily{{\tiny$\blacktriangleright$~\scriptsize#2}}\end{Spacing}} %
}
%
% --- SET UP TITLE ---
%
\setlength{\droptitle}{0.0in}
\backmatter
\pretitle{\noindent\huge\sffamily Course Syllabus \LARGE\par\noindent} 
\posttitle{\par\vskip 2.0em}
\preauthor{}
\postauthor{\par}
\predate{}
\postdate{\noindent\rule{\linewidth}{0.3pt}}
\title{Chemistry 111 / Spring 2016}
\date{}
\author{}
%----------------------------------------------------------------------------s
%
% --- BEGIN DOCUMENT ---
%
\begin{document}
\setsecnumdepth{subsubsection}
\maketitle
%\setsecnumdepth{subsection}
\thispagestyle{courseinformationtitle}
%
% --- INSTRUCTOR ---
%
\section{Instructor}
\begin{tabular}{rl|rl}
Name: & Dr. Dale J. Brugh & Office Phone: & 740-368-3530 \\
Email: & \href{mailto:djbrugh@owu.edu}{djbrugh@owu.edu} & Cell Phone: & 614-746-2397 \\
Office: & SCSC 262 & Office Hours: & \href{https://dephlogisticated.net/drbrugh/officehours/}{Posted on Website} \\
\end{tabular}
%
% --- MEETINGS ---
%
\section{Meetings}
\begin{tabular}{crcrl}
MWF & 8:00 am & to & 8:50 am & SCSC 244 \\
R & 12:10 pm & to & 1:00 pm & SCSC 244
\end{tabular}
\marginnote{-0.2in}{Follow-up quizzes and exams are given at 6:30 p.m. on Tuesday evenings.}
%
% --- WEBSITE ---
%
\section{Website}
The course website is located at \href{https://dephlogisticated.net/genchem}{dephlogisticated.net/genchem}. The website is an important extension of this syllabus and should be read carefully. 
%
% --- PREREQUISITES ---
%
\section{Prerequisites}
A grade of $\mathrm{C}-$ or better in CHEM110 is required to take this course. Proficiency in arithmetic and algebra is required and assumed throughout the course. Calculus is not required. Background in high school physics and chemistry will be helpful. 
%
% --- LABORATORY ---
%
\section{Laboratory}
Registration in a laboratory section is required. 
%
% --- MATERIALS ---
%
\section{Materials}
These items are required. Additional details, including purchasing options, can be found on the course website at \href{https://dephlogisticated.net/genchem-materials}{dephlogisticated.net/genchem-materials}.
\begin{enumerate}
\item \textit{General Chemistry: The Central Science} by Brown, LeMay, Bursten, Murphy, and Woodward, Twelfth Edition, Pearson, 2012. ISBN: 9780321696724. 
\item Access to the Sapling Learning course with the title \\ Ohio Wesleyan University - CHEM 111 - Spring16 - BRUGH
\item Scientific, non-programmable, non-graphing calculator such as the Texas Instruments TI-30Xa or the  Casio FX260SLRSC.
\end{enumerate}
%
% --- CONTENT ---
%
\section{Content}
This course is a survey of the principles of chemistry, including solutions, kinetics, equilibrium systems, thermodynamics, electrochemistry, and coordination compounds. The course topics are listed in Table~\ref{tab:topics}. 
%
% --- Table: Course Topics ---
%
\begin{table}[h]
\caption{\sffamily Topics covered in Chemistry 111}
\label{tab:topics}
\renewcommand{\arraystretch}{1}
\begin{tabular}{l|l|l} \toprule
Properties of Solutions & Acid-Base Equilibrium    & Thermodynamics \\
Colligative Properties  & Common Ion Effect        & Entropy \\
Rates of Reactions      & Buffers                  & Gibbs Free Energy \\
Integrated Rate Laws    & Titration                & Electrochemistry \\
Reaction Mechanisms     & Solubility Equilibrium   & Coordination Compounds \\
Chemical Equilibrium    & Complex Ions             & Crystal Field Theory \\
\bottomrule
\end{tabular}
\end{table}
\marginnote{-0.65in}{Topics are not necessarily covered in this order, and not all topics are covered with equal depth.} 
%
% --- GOALS ---
%
\section{Goals}
Chemistry provides a microscopic view of the Universe that is applicable to any discipline. Phenomena as different as the ozone hole over the South Pole and Earth's magnetism are given common ground in chemistry where the properties and interactions of atoms and molecules give rise to all observed macroscopic phenomena. Chemistry makes the Universe understandable and interpretable. 

I want to share some of this with you. During this course I want to 
\begin{inparaenum}[\bfseries (a\upshape)]
\item provide you with the tools necessary to form a realistic mental image of the microscopic world of atoms and molecules,
\item provide you with the tools necessary to understand what a chemist does,
\item provide you with the tools to continue your education in chemistry,
\item develop your ability to analyze and interpret experimental results from a chemist's point of view,
\item help you understand how we know what we know, and
\item improve your problem solving skills.
\end{inparaenum}

Another important goal of the course is to improve your study skills. I want you to learn to be efficient and effective in your studying so that you can get more done in less time. By the end of the course, I would like academic success to be an automatic part of your life, not something you have to struggle to achieve. 
%
% --- LEARNING OBJECTIVES ---
%
\section{Learning Objectives}
Learning objectives are the things you should be able to do at the end of the course. For each topic covered in this course, you are provided with a detailed list of learning objectives called Be Able Tos, or BATS for short. A complete list of detailed BATS for each course topic can be found at \href{https://dephlogisticated.net/chem111/lecture/objectives}{dephlogisticated.net/genchem-objectives}. These BATS are very detailed (granular), and it might be difficult to see the overall objectives of the course from them. 

A higher level (less granular) list of learning objectives might be helpful. At the end of the course, you should be able to 
\begin{inparaenum}[\bfseries (a\upshape)]
\item use colligative properties to predict characteristics of solutions;
\item determine the rate of a reaction and its dependence on concentration, time, temperature, and pressure;
\item compute equilibrium concentrations for all species in a reaction;
%\item predict how equilibrium concentrations will be altered by a change in reaction conditions;
\item apply equilibrium to solutions of acids, bases, ionic compounds, and complex ions;
\item compute changes in entropy, enthalpy, and Gibbs free energy for reactions;
\item explain entropy's role in driving chemical and physical changes;
%\item predict equilibrium constants based on thermodynamic data and visa-versa; 
\item explain and predict the operation of galvanic and electrolytic cells;
\item describe the properties of transition metal coordination compounds using crystal field theory, and
\item name transition metal coordination compounds. 
\end{inparaenum}
%
% ---TIME REQUIREMENT ---
%
\section{Time Requirement}
Each \marginnote{-0.1in}{The time required will depend on your study skills. Estimates are for a typical student.} of the four weekly course meetings is 50 minutes in length, requiring a total of $3.33$ hours per week. For each meeting you can expect to spend about $2$ to $3$ hours outside of class reading, studying, and working assigned exercises and quizzes.
%
% --- WEEKLY ROUTINE ---
%
\section{Weekly Routine}
This course is primarily lecture-based. Each lecture meeting has a topic and an assigned reading which can be found on the course website. You prepare for lecture by reading the assigned reading. A homework assignment on Sapling Learning is due each Friday. There will be four quizzes and three exams during the semester. 
%
% --- THINGS I GRADE ---
%
\section{Things I Grade}
You and I determine your progress in this course using the scores derived from evaluating the quality and accuracy of your answers to questions posed in homework, quizzes, and exams. This section provides details for each of these.
%
% --- Homework ---
%
\subsection{Homework} 
Homework is assigned and due each Friday on the Sapling Learning website. Questions are added to each homework assignment as the week progresses. Please check the Sapling Learning site every night for homework updates. Each weekly assignment is worth $100$ points. Your lowest one (1) homework score is dropped before calculating your course score. 
%
% --- Quizzes ---
%
\subsection{Quizzes}
Dates for quizzes are listed in the course schedule on the course website. Each is worth 25 points. They are administered at the start of a normal class meeting and typically require 20 minutes to complete. The material covered on each quiz is listed on the course website. Solutions for quizzes are posted on the course website after the quizzes are graded. \marginnote{-0.1in}{Scores on follow-up quizzes replace original quiz scores only if they are higher than the original.} A practice quiz is available before each quiz. You may take a follow-up quiz to replace the score on the original quiz. Follow-up quizzes are given at 6:30 p.m. the Tuesday after the original quiz. 
%
% --- Exams ---
%
\subsection{Exams}
Dates of exams are listed in the course schedule on the course website. Each is worth 100 points. Three exams are given during the semester. They are administered starting at 7:30 a.m.\ on the scheduled date. The material covered on each exam is listed on the course website. Solutions are posted on the course website after all exams have been graded. \marginnote{-0.25in}{Scores on follow-up exams replace original exam scores only if they are higher than the original.} A practice exam is available for each exam. You may take a follow-up exam to replace the score on the original exam. Follow-up exams are given at 6:30 p.m. the Tuesday after the original exam. 
%
% --- Final Exam ---
%
\subsection{Final Exam}
The \marginnote{-0.1in}{The final exam is given at the time specified by the Registrar. I have no control over this time.} final exam is three hours in length, and it is cumulative over the entire semester. No solutions to the final exam are posted, and you are not allowed to take your final exam with you after I grade it. You may review the grading by making an appointment with me, but you may not take possession of the exam.
%
% --- LABORATORY SCORES ---
%
\section{Laboratory Scores}
Your \marginnote{-0.1in}{I do not adjust laboratory scores. What you earn as a percentage is what will be included in your course score.} laboratory instructor is responsible for evaluating your laboratory work. He or she will assign a score to your laboratory work, and at the end of the course I will convert your laboratory score to a percentage that is included in the computation of your course score.
%
% --- COURSE SCORE ---
%
\section{Course Score}
Your course score is a weighted average of the scores you earn on exams, quizzes, homework, and laboratory work. These scores are weighted according to the percentages shown in Table~\ref{tab:weights}. Before calculating your course score, \marginnote{-0.1in}{Your lowest 1 homework score is dropped.} I will throw away your lowest one (1) homework score.  
%
% --- Table: Evaluation Weights
%
\renewcommand{\arraystretch}{1.15}
\begin{table}[h]
\caption{\sffamily Weight of evaluated items}
\label{tab:weights}
\begin{tabular}{rc|rc}
\hline\hline
\textbf{Evaluation Item} & \textbf{Weight} & \textbf{Evaluation Item} & \textbf{Weight} \\ \hline
Homework & 20\% & Exams & 20\% \\
Laboratory & 20\% & Final Exam & 20\% \\
Quizzes & 20\% &  \\ 
\hline\hline
\end{tabular}
\end{table}
\section{Letter Grade}
Letter grades are assigned at the end of the course according to the minimum course score requirements listed in Table~\ref{tab:lettergrades}. Course scores below $55\%$ are considered failing. Please see \href{https://dephlogisticated.net/lettergrades}{dephlogisticated.net/lettergrades} for more detail about how your course letter grade is determined. 

\begin{table}[h]
\caption{\sffamily Minimum course scores necessary for each letter grade.}
\label{tab:lettergrades}
\begin{tabular}{cl||cl} \toprule
\textbf{Minimum Score} & \textbf{Letter Grade} & \textbf{Minimum Score} & \textbf{Letter Grade} \\ \hline
97 & \hspace{0.3in}A$+$ & 72 & \hspace{0.3in}C$+$ \\
88 & \hspace{0.3in}A & 68 & \hspace{0.3in}C \\
85 & \hspace{0.3in}A$-$ & 65 & \hspace{0.3in}C$-$ \\
82 & \hspace{0.3in}B$+$ & 62 & \hspace{0.3in}D$+$ \\
78 & \hspace{0.3in}B & 58 & \hspace{0.3in}D \\
75 & \hspace{0.3in}B$-$ & 55 & \hspace{0.3in}D$-$ \\
\bottomrule
\end{tabular}
\end{table}
%
% --- ADDITIONAL INFORMATION ---
%
\section{Additional Information}
Please \marginnote{-0.1in}{The course website is an extension of this syllabus. It is essential that you read the material contained on the course website.} see the course website at \href{https://dephlogisticated.net/genchem}{dephlogisticated.net/genchem} for additional information such as the detailed course schedule, assignments and due dates, assignment solutions, office hours, granular course objectives, suggestions for success, my expectations of you, grade advice, and policies on assignment of letter grades. The website also contains detailed policies on attendance, late submission of work, email communication, disability accommodations, and classroom behavior. All material found on the course website is an extension of this syllabus and constitutes my contract with you for this course. 
%
% --- LICENSE AND SOURCE CODE ---
%
\section{License and Source Code}
\copyright\ 2016 by Dale J. Brugh. Course Syllabus for Chemistry 111 (2016) is made available under a Creative Commons Attribution-ShareAlike 4.0 International License (CC BY-SA 4.0). See \href{https://creativecommons.org/licenses/by-sa/4.0/}{https://creativecommons.org/licenses/by-sa/4.0/} for details about your rights under this license. The \LaTeX\ source code is available under the same license from Github at \href{https://github.com/djbrugh/chem111}{https://github.com/djbrugh/chem111}. \marginpar{\vspace{-0.85in}\hfill\noindent\href{https://creativecommons.org/licenses/by-sa/4.0/}{\includegraphics[width=1in]{figs/cc-by-sa.pdf}}}
%
% --- END DOCUMENT ---
%
\end{document}
